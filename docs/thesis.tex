\documentclass[12pt]{report}
\usepackage[T1]{fontenc}
\usepackage[utf8]{inputenc}
\usepackage{graphicx}
\usepackage{amsmath,amssymb,amsfonts}
\usepackage{txfonts}

%\usepackage[polish]{babel}

\renewcommand{\chaptername}{Rozdział}
\renewcommand{\contentsname}{Spis treści}
\renewcommand{\figurename}{Rys.}
\renewcommand{\tablename}{Tab.}
\renewcommand{\listfigurename}{Spis rysunków}
\renewcommand{\listtablename}{Spis tabel}
\renewcommand{\bibname}{Bibliografia}

\pagestyle{headings}

\setlength{\textwidth}{14cm}
\setlength{\textheight}{20cm}

\newtheorem{definition}{Definicja} % przykład nowego środowiska 
\newtheorem{example}{Przykład}[chapter] % przykład nowego środowiska 
\newtheorem{corollary}{Wniosek}[chapter] % przykład nowego środowiska 



\begin{document}
\tableofcontents	% generuje spis treści ze stronami !!!

\chapter{Wstęp} \label{rozdz.wstep} 
{\em Praca MUSI stanowić samodzielne opracowanie przez dyplomanta WYBRANEGO TE\-MATU
BADAWCZEGO pod kierunkiem promotora. 
Temat i zakres pracy powinien wiązać się ze specjalnością, na której studiuje
dyplomant.\\
\indent Orientacyjna objętość pracy inżynierskiej/licencjackiej (I-go stopnia)
to 50-80 stron, zaś pracy magisterskiej (II-go stopnia) -- 70-120 stron.}

\section{Problematyka i zakres pracy}
Niniejsza praca dotyczy zakresu (inżynierii oprogramowania/sieci
komputero\-wych/gra\-fiki komputerowej/sztucznej inteligencji/algorytmów ewolucyjnych/technologii baz danych). ....\\
\indent Głównym przedmiotem/celem pracy jest
stworzenie/opracowanie/przeanalizo\-wanie/zaprojektowanie.... {\bf ten fragment
pracy zawierać musi wyraźne określenie problemu badawczego oraz jego pogłębioną
analizę}\\

\indent Dlaczego podejmowanie tej tematyki jest potrzebne? Czy są inne
rozwiązania tego problemu/tych problemów? Jakie? czy są lepsze/gorsze,
tańsze/droższe, itp.
\indent Przed jakimi wyzwaniami stoi osoba podejmująca tematykę? \\
\indent {\bf Określić spodziewane efekty pracy:} W wyniku doświadczeń przeprowadzonych w zakresie pracy polepszeniu
uległo.... podać konkretne wskaźniki rezultaty, jak np. przyspieszenie obliczeń,
redukcja kosztu, nowe oprogramowanie itp. w złotówkach, sekundach, procentach,
roboczogodzinach itp. 

{\bf (razem max. 3 strony - strona przeliczeniowa = 1800 znaków, średnio 30
wierszy po 60 znaków)}

\section{Cele pracy}
Wymienić w punktach cele pracy, rozpoczynając od celów poznawczych (dotyczą\-cych
zebrania wiadomości, przybliżenia/popularyzacji technik / metod / zagaadnień).
W~drugiej kolejności wyienić cele praktyczne.
\subsubsection{Przybliżenie (popularyzacja) metody/technologii/systemu...}
\subsubsection{Propozycja rozwiązania problemu....}
\subsubsection{Opis zastosowania technologii X w problemie Y...}
\subsubsection{Przedstawienie prototypu systemu/układu/aplikacji...}
\subsubsection{Określenie przydatności algorytmu Z do rozwiązania problemu T.....
}
\subsubsection{Opracowanie strategii ... w celu poprawy wydajności/jakości...}
\subsubsection{Ocena możliwości wdrożenia proponowanych rozwiązań, ich wartość
praktyczna, lokalne i globalne możliwości zastosowania}
\subsubsection{itp.}

Każdy cel opisać w minimum 2-3 zdaniach. Użyte określenia muszą być powszechnie
zrozumiałe, nie stosujemy skrótów, slangu,  tzw. makaronizmów, np. ,,softłer''.
{\bf cały podrozdział ok. 1 strony przeliczeniowej czyli 1800 znaków}

\section{Metoda badawcza}
\begin{itemize}
\item Studia literaturowe
\item Analiza budowy i działania istniejących produktów
\item Projektowanie i prototypowanie nowatorskich rozwiązań 
\item Obliczenia i ........
\end{itemize}

Każdy element opisać w minimum 2-3 zdaniach. Np. studia literaturowe powinny
odnosić się do charakterystyki wykorzystanych źródeł książkowych, czyli: Jaka
jest podstawowa literatura dziedziny, czy jest dostępna w języku polskim, czy trzeba je tłumaczyć, czy wiedza na ten
temat jest zebrana w jednym miejscu, czy jej synteza jest osobnym zadaniem itp. 
Jak duży jest udział źródeł elektronicznych w tej ,,działce'' wiedzy i badań,
itd. \\
\indent Jakie metody badawcze są typowe dla danego tematu. Dlaczego je
zastosowano, ewentualnie dlaczego zastosowano inne? \\
WYMAGANE ODNOŚNIKI DO POZYCJI BILIOGRAFII.\\
{\bf cały podrozdział ok. 1 strony przeliczeniowej czyli 1800 znaków}.

\section{Przegląd literatury w dziedzinie}
Rozszerzyć odpowiedni podpunkt z metody badawczej, np. wg podziału:
\subsubsection{Źródła książkowe polskojęzyczne i tłumaczenia}
\subsubsection{Źródła książkowe obcojęzyczne}
\subsubsection{Artykuły naukowe, raporty z badań, komunikaty konferencyjne,
dokumentacje techniczne, manuale, instrukcje}
\subsubsection{Źródła elektroniczne}


\section{Układ pracy}
Tematem pracy jest: ....., zaś za główny cel przyjęto ...... . \\
Rozdział \label{rozdz.wstep} zawiera wstęp i cele pracy. W rozdziale drugim
opisano/...... w Rozdziale 3. zawarto............ Rozdział 4. przedstawia..... \\
W podsumowaniu pracy przedstawiono..........................., z czego wynika,
że ................  \\
Najważniejszym wnioskiem/wynikiem/rezultatem pracy jest..................\\ {\bf wyraźnie określić
CO TO JEST}. \\

{\bf cały podrozdział ok. 1 strony}.




\chapter{Tytuł części teoretycznej} \label{etykietarozdzialu2}
\section{Podstawowe definicje}
Ten podrozdział powinien zawierać dokładny opis terminologii  pojęć zasadniczych dla tematu pracy, którymi autor będzie się posługiwał przy realizacji głównych celów pracy. 


\section{Istniejące rozwiązania w dziedzinie}
W tym podrozdziale zostaną opisane.....
\subsection{Sprzęt}
.........................
\subsection{Oprogramowanie i wdrożone systemy}
.....................................
\subsection{}
...................

\section{Wady i słabe punkty istniejących rozwiązań}
\subsection{Efektywność}
..........................
\subsection{Utrudniony dostęp}
..............
\subsection{Wysokie koszty}
...............................

\chapter{Dalsze uwagi o edycji i~formatowaniu pracy}
Pracę w \LaTeX'u najlepiej składać w szablonie {\tt report}, ze względu na jendostronny wydruk (jak w {\tt article}) i możliwość dzielenia pracy na rozdziały, a co za tym idzie, tworzenia spisu treści, spisu tabel, rysunków. 
\begin{example}
Przyklad
\end{example}

\begin{corollary}
Wniosek
\end{corollary}
\section{Bibliografia i przypisy}
Spis litertury dołącza się  w \LaTeX'u automatycznie na końcu pracy (zob.
komenda {\tt begin{thebibliography}}). Informacje o sposobie cytowania zawarte
są na stronie Bibilioteki Głównej PŁ\\ także udostępnione na
\underline{\tt http://ics.p.lodz.pl/\textasciitilde aniewiadomski}. \\

\indent Przykład cytowania --  jak podaje praca \cite{kacprzyk86}, ......,
jednakże autorzy [2] twierdzą, iż.....\\


\indent Za każdym razem, kiedy w pracy pojawia się treść na podstawie jakiegoś
tekstu źródłowego czyjegoś autorstwa, oznaczamy takie miejsce
przypisem\footnote{Treść przypisu pierwszego}. Przypis zawierać musi  numer
jakim w spisie literatury, czyli bibliografii, oznaczono tę pracę, np.
tak\footnote{[3], ss. 3--6 (czyli praca trzecia w spisie literatury,
wykorzystany fragment znajduje się na stronach od 3. do 6.)}. {\bf
Wszystkie źródła tekstów, rysunków, danych, wykresów, schematów, kodów i
informacji wykorzystanych w pracy muszą być zamieszczone w bibliografii.
Wszystkie pozycje literatury zamieszczone w bibliografii muszą być cytowane w
treści pracy, na dowód, iż zostały rzeczywiście użyte przy pisaniu pracy.}

\subsubsection{Źródła elektroniczne}
Źródła elektroniczne, zwłaszcza internetowe należy cytować z należytą uwagą na
ich jakość. Nie cytujemy źródeł wątpliwej jakości lub wtórnie przekazujących
czy też powielających wiedzę zawartą w innych źródłach, np. fora internetowe lub
wikipedia.\\
\indent {\bf Wszystkie wykorzystane źródła elektroniczne powinny być przez Autora
pracy skopiowane \underline{w dniu ich wykorzystania} i
dołączone np. na CD/DVD do wersji drukowanej pracy.\\
\indent Odnośniki do źródeł elektronicznych muszą zawierać pełną ścieżkę, np. do
pliku lub rysunku, a nie jedynie domenowy adres portalu, np.\newline {\tt
http://serwer.com/temat/podtemat/katalog/plik\_strony.html} (stan na dzień:
2009-12-05)\\
ale nie\\
{\tt www.portal.pl}.} (!!!!!)

Niedochowanie tego wymogu może stać się powodem odrzucenia pracy ze wzglę\-dów
formalnych (,,brak możliwości weryfikacji źródeł wykorzystanych w pracy''). 

\section{Polskie akapity, cudzysłowy, itp.}
Akapity stosujemy zawsze z wcięciem, ale bez wiersza odstępu pomiędzy akapitami. 
Ta forma jest przyjęta dla publikacji polskojęzycznych. {\bf W szczególnych
przypadkach (także w tym szablonie)} akapit występujący bezpośrednio po tytule
rozdziału, sekcji, podsekcji itp. NIE JEST WCIĘTY.\\
\indent Ten akapit JEST WCIĘTY. NIE MA także PUSTEGO WIERSZA pomiędzy tym
akapitem a poprzednim. \\
\indent Podobne uwagi dotyczą wszystkich innych elementów formatowania pracy --
muszą być zgodne ze zwyczajami przyjętymi W JĘZYKU POLSKIM.	Np. cu\-dzysłowy
wyglądają tak: ,,cudzysłów'', ale nie ''cudzysłów'', albo też `cudzysłów' czy
``cudzysłów''. 



\section{Definicje i wyrażenia matematyczne}
\begin{definition} \label{def.definicja1}
Niech $\cal X$ będzie przestrzenią.....
\end{definition}

Do definicji odnieść sie można poprzez jej etykietę: jak podano w Def.~\ref{def.definicja1}

Przykładowe podkreślenie... \underline{tekst podkreślony}, pogrubienie: {\bf
tekst pogrubiony} oraz wyrożnienie {\em tekst wyróżniony, czyli kursywa}. 
Dalszy tekst rozdziału
Dalszy tekst rozdziału
Dalszy tekst rozdziału
Dalszy tekst rozdziału
Dalszy tekst rozdziału
Dalszy tekst rozdziału
Dalszy tekst rozdziału
Dalszy tekst rozdziału
Dalszy tekst rozdziału a teraz koniec linii... \\
\indent ... i nowy akapit. Akapity muszą być standardowo wcięte.  


Przykład wzoru matematycznego numerowanego

\begin{equation} \label{wzoreinsteina}
E=m\cdot c^2
\end{equation}

{\bf Wszystkie symbole matematyczne występujące w tekście ,,na bieżąco'',
czyli nieoznaczone numerem równania TAKŻE PISZEMY W TRYBIE MATEMA\-TYCZNYM, CZYLI
K U R S Y W Ą} : $a=b\cdot c$, ale nie: a= b*c (!!)\\

\indent Numeracja wzoru -- ZAWSZE w POSTACI (\#.\#\#)
Jak podaje wzor (\ref{wzoreinsteina}).... (koniec linii). \\
\indent Wyrażenia matematyczne można też wpisywać w wierszu -- używamy wów\-czas znaku '\$', który rozpoczyna i kończy wyrażenie, np. wg Einsteina $E=m\cdot c^2$...


\section{Jak wstawiać rysunki? tabele? }
A teraz pora na rysunek:
\begin{figure}[!t]
\centering
%\includegraphics[width=7cm]{1figmftall} 
\caption{Funkcja przynależności zbioru rozmytego -- Podpis ZAWSZE POD rysunkiem,
numeracja w postaci \#.\#\#. } (wypada podać źródło, czyli literaturę,
z której rysunek pochodzi, ewentualnie {\em opracowanie własne}.)
\label{fig.funkcja.przyn}
\end{figure}


\begin{table}[!t]
\centering
\caption{Tytuł tabeli ZAWSZE NAD TABELĄ, numeracja w formie \#.\#\#. (wypada podać źródło, czyli literaturę,
z której tabela pochodzi, ewentualnie {\em opracowanie własne}.)} 

\label{tabls1}

{\footnotesize 
\vspace{5mm}
\begin{tabular}{c c c c c}
\hline\noalign{\smallskip}
{\bf Alg.} & {\bf tytuł kolumny 1} & {\bf tytuł kolumny 1} & {\bf Tytuł kolumny
3} & {\bf ....}     \\

\hline\noalign{\smallskip}
a & b & c & d & e  \vspace{3mm} \\ 
\noalign{\smallskip}
 a & b & c & d & e \\

\noalign{\smallskip}
%%%
\end{tabular}
}
\end{table}


Rysunki i tabele nie powinny przekraczać 0.9 szerokości tekstu i zasadniczo
powinny występować na górze strony. \\

\indent Odnosić się do rysunku można poprzez jego etykietę ''label'', np. jak widać na rys. \ref{fig.funkcja.przyn}......

Jak widać, rysunek nie wypada w dokumencie w tym samym miejscu co w kodzie, choć czasem się tak zdarza. Jeśli potrzebujesz przenieść rysunek, zajrzyj do rozdzialu 2.11. manuala pt. {\em Wstawki}. 

\section{Listy wypunktowana i numerowana}

\begin{itemize}
\item pierwszy element listy wypunktowanej
\item drugi...
\item trzeci...
\end{itemize}


Nowy akapit z lista numerowaną. 
\begin{enumerate}
 \item pierwszy element listy NUMEROWANEJ
 \item drugi...
 \item trzeci...
 \item trzeci...
 \item trzeci...
 \end{enumerate}

\section{Przenoszenie wyrazów}
Skorzystaj z polecenia {\tt hyphenation}\\ w preambule dokumentu, lub dziel
wyrazy ,,ręcznie'' czyli właśnie tak jak tu: po\-dzie\-lo\-ne wy\-ra\-zy. 

\chapter[Technologie i metody użyte...]{Technologie i metody użyte w~części
badawczej}

{\em Tytuł tego rozdziału ma dwie wersje: zwykłą, (w kodzie: w nawiasach
klamrowych), która
pokazuje sie na stronie rozpoczynającej rozdział, oraz krótką (w kodzie: w nawiasach
kwadratowych), która pokazuje sie w spisie treści i w nagłówku}

W rozdziale \ref{etykietarozdzialu2} podano podstawy teoretyczne i ogólny zakres
pracy. W niniejszym rozdziale opisana zostanie technologia XYZ oraz metoda ABC
użyta w części praktycznej, patrz rozdział~\ref{rozdz.czesc.prakt}. 

\section{Sprzęt}
...................
\subsection{Element 1}
.........................
\subsection{Element 2}
......................

\section{Oprogramowanie}
..........................
\subsection{Serwer baz danych}
........................
\subsection{Środowisko zintegrowane}
..........................
\subsection{Oprogramowanie klienckie}

\section{Technologie i metodologie programistyczne}
..................
\subsection{Język programowania}
......................
\subsection{Biblioteki}
.......................
\subsection{Wzorce projektowe}
.......................

\section{Inne, np. narzędzia i metody symulacji, }

\chapter{Aplikacja/system/projekt "XYZ"} \label{rozdz.czesc.prakt}
Ta część pracy może być podzielona na więcej rozdziałów, np kiedy autor chce
w~szczególności podkreślić któryś z etapów projektu. W zależności od tematu i~celów pracy, pewne sekcje można dodać (np. przy projektowaniu sieci, instalacji
i~konfiguracji serwerów usług sieciowych), inne zaś pominąć.

\section{Analiza wymagań}
\subsection{Studium możliwości}
\subsection{Wymagania funkcjonalne}
.................
\subsection{Ograniczenia projektu}

\section{Projekt}
\subsection{Projekt warstwy danych}

\begin{enumerate}
\item normalizacje baz danych
\item projekt bazy/baz 
\item grupy użytkowników i ich prawa dostępu do danych (zależne od implementacji bazy)
\item ew. diagramy klas warstwy danych
\end{enumerate}
\subsection{Projekt warstwy logiki}
\begin{enumerate}
\item Diagramy i scenariusze przypadków użycia
\item Diagramy przepływu danych (lub ich odpowiedniki)
\item ew. diagramy klas, wzorce projektowe itp.
\end{enumerate}

\subsection{Projekt warstwy interfejsu użytkownika}
\subsubsection{Wybór środowiska i platformy działania}
\subsubsection{Rodzaj aplikacji (klient-serwer, thick/thin client, aplikacja
,,biurkowa'', usługa, klient hybrydowy, itp.}
\subsubsection{Technologie projektowania i realizacji interfejsu użytkownika,
np. biblioteki}


\section{Implementacja: punkty kluczowe}

\section{Testy i wdrożenie}
\subsection{Testy wydajności}
\subsection{Testy regresyjne}
\subsection{Testy bezpieczeństwa}
\subsection{Dalsze testy}
\subsection{Testy...}

\section{Konserwacja i inżynieria wtórna}
Jak przebiega eksploatacja systemu/projektu? Jakie wady i zalety ujawniły się po
np. 2-miesięcznym okresie testowania i użytkowania? \\
\indent Jak można skorzystać z tej wiedzy praktycznej pod kątem roz\-bu\-do\-wy pracy? Jakie elementy systemu powinny zostać w pierwszej kolejności zmodyfikowane?  

\chapter{Podsumowanie}
\section{Dyskusja wyników}
Dzięki zrealizowaniu pracy poprawie uległa wydajność ....... Ponadto, o ?? \%
skrócony został czas ........, a koszty osiągnięcia zamierzonego efektu zostały
zmniejszone z ???pln do ???pln za godzinę/ dzień/ jednostkę sprzętu.........\\
\indent Które cele pracy udało sie zrealizować? co z tego wynika? Które cele
pracy pozostały niezrealizowane i dlaczego? 

\section[Ocena możliwości wdrożenia...]{Ocena możliwości wdrożenia proponowanych
\newline rozwiązań...}
... ich wartość praktyczna, lokalne i globalne możliwości zastosowania, kwestia
praw autorskich do powstałych produktów, itp. 

\section{Perspektywy dalszych badań w dziedzinie}
Jak można kontynuować tę pracę, zwłaszcza pod kątem studiów
uzupełniających magisterskich i/lub doktoranckich. Co jeszcze powinno być
zrobione lub ulepszone? Co należy zmienić lub poprawić w pracy z dzisiejszego punktu widzenia?


\addcontentsline{toc}{chapter}{Bibliografia} 
\begin{thebibliography}{99}
\bibitem{kacprzyk86}
Kacprzyk J. (1986) Fuzzy sets in system analysis.  PWN, Warsaw (in Polish).
\bibitem{kacprzyk99b}
Kacprzyk J., Strykowski P. (1999) Linguistic Data Summaries for Intelligent Decision Support, Proceedings of EFDAN'99. 4-th European Workshop on Fuzzy Decision Analysis and Recognition Technology for Management, Planning and Optimization, Dortmund, 1999, 3--12.
\bibitem{kacprzyk01d}
Kacprzyk J., Yager R. R. (2001) Linguistic summaries of data using fuzzy logic. International Journal of General Systems 30:133--154 

\end{thebibliography}

\addcontentsline{toc}{chapter}{Spis rysunków} 
\listoffigures

\addcontentsline{toc}{chapter}{Spis tabel} 
\listoftables


\addcontentsline{toc}{chapter}{Załączniki} 
\chapter*{Załączniki}
\begin{enumerate}
\item Załącznik nr 1
\item Załącznik nr 2
\item Załącznik nr 3
\end{enumerate}


\end{document}
